
\subsection{Curvature}

\begin{definition}
  Let \(\gamma\) be a unit-speed curve with parameter \(t\), the
  \textit{curvature}\index{Curvature} of \(\gamma\) at point \(\gamma(t)\) is given by
  \[
\kappa(t) = \norm{\gamma''(t)}
\]
\end{definition}

\begin{example}
  Consdier the circle with center \((x_{0},y_{0})\) and radius \(r\). A unit-speed
  parametrization is given by
  \[
\gamma(t) = (x_{0} + r\cos \frac{t}{r}, y_{0} + r\sin \frac{t}{r})
\]
  and \(\gamma\) is unit speed since
  \[
\norm{\gamma'(t)} = \sqrt{(-\sin^{2}\frac{t}{r})^{2} + (\cos \frac{t}{r})^{2}
} = 1
\]
  Then the curvature of a circle is given by
  \[
\kappa(t) = \norm{\gamma''(t)} = \sqrt{(-\frac{1}{r}\cos\frac{t}{r})^{2} + (-\frac{1}{r}\sin \frac{t}{r})^{2}}
      = \frac{1}{r}
\]
  Intuitively, this makes sense since the curvature is the reciprocal of the
  radius, this means small circles should have large curvature and large circles
  should have small curvature as expected.
\end{example}

As dicsovered previously, every curve has a unit-speed reparemtrization but it
is not always convenient or possible to express it in terms of elementary
functions. The following proposition gives a way to express curvature in terms of
\(\gamma\) itself instead of a unit-speed reparametrization of it.

\begin{proposition}
  Let \(\gamma(t)\) be a regular curve in \(\mathbb{R}^{3}\). Then its curvature is given by
  \[
\kappa = \frac{\norm{\gamma'' \times \gamma'}}{\norm{\gamma'}^{3}}
\]
  where \(\times\) denotes the cross product.

  \begin{proof}
    The proof of this proposition is computationally intensive so is omitted.
    The basic idea is to consider a unit-speed parameter \(s\) of \(\gamma\). Then apply
    the chain rule to \(\gamma\) and plug it into the unit-speed curvature formula and
    perform some more differentiating and vector products to obtain \(\kappa\) in terms
    of just \(\gamma\) and its derivatives.
  \end{proof}
\end{proposition}


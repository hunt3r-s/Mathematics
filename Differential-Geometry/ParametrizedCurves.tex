\subsection{Parameterized Curves}


\begin{definition}
		A \textit{parameterized curve}\index{Parametrized curve} in \(\mathbb{R}^{n}\) is a differentiable map
            \[
\gamma \colon \intoo{\alpha, \beta} \rightarrow \mathbb{R}^{n}
\]
		of an open interval \(\intoo{\alpha, \beta} = \{t \in \mathbb{R} \mid \alpha < t < \beta \}\) onto \(\mathbb{R}^{n}\) where \(-\infty \leq \alpha < \beta \leq \infty\)

\end{definition}

	The above definition means that \(\gamma\) is a correspondence from each \(t \in \intoo{\alpha, \beta}\) to a point \(\gamma(t) = ((\gamma_{1}(t), \gamma_{2}(t), \ldots, \gamma_{n}(t)) \in \mathbb{R}^{n}\) such that each function \(\gamma_{1}(t), \gamma_{2}(t), \ldots, \gamma_{n}(t)\) is differentiable.


\begin{definition}
		A \textit{level curve}\index{Level curve} is a parametrized curve in \(\mathbb{R}^{3}\) such that it can be defined by the pair of equations

\[
f_{1}(x, y, z) = c_{1}, \text{and} f_{2}(x, y, z) = c_{2}
\]
		In other words, a level curve is the set of all points in the domain of \(f\) that reach a certain value \(c\).

\end{definition}


\begin{example}
		Consider the function \(y = x^{2}\), we seek to find a parametrization \(\gamma(t)\) of the function. That means the parametrization must satisfy \(\gamma_{2}(t) = \gamma_{1}(t)^{2}\) for all \(t \in (\alpha, \beta)\) where \(\gamma\) is defined.
		An obvious solution to the equation \(\gamma_{2}(t) = \gamma_{1}(t)^{2}\) is \(\gamma_{1}(t) = t, \gamma_{2}(t) = t^{2}\). Since the \(x\) coordinate of \(\gamma(t)\) is just \(t\), and \(x\) can take on any value in a parabola, we know that the domain of the parametrized curve is all real numbers. Using this observation and our solution to \(\gamma_{2}(t) = \gamma_{1}(t)^{2}\), we obtain

\[
\gamma \colon \intoo{-\infty, \infty} \rightarrow \mathbb{R}^{2}, \;\; \gamma(t) = (t,t^{2})
\]

\begin{remark}
			The equation \(\gamma(t) = (2t, 4t^{2})\) is also a valid parametrization, as are infinitely many other pairs of coordinates that satisfy the equation.

\end{remark}

\end{example}


\begin{example}
		Now consider the circle \(x^{2} + y^{2} = 1\). An obvious solution to the equation

\[
\gamma_{1}(t)^{2} + \gamma_{2}(t)^{2} = 1
\]
		is

\[
\gamma_{1}(t) = \cos^{2}(t), \gamma_{2} = \sin^{2}(t)
\]
		since

\[
\gamma_{1}(t)^{2} + \gamma_{2}(t)^{2} = 1
\]
, for all values of \(t\), the domain of \(\gamma\) will be all real numbers. Thus, we get the curve

\[
\gamma \colon \intoo{-\infty, \infty} \rightarrow \mathbb{R}^{2}, \;\; \gamma(t) = (\cos^{2}(t), \sin^{2}(t))
\]

\begin{remark}
			Taking \(x = t\) instead of \(x = \cos^{2}(t)\) would have yielded the parametrization of \(\gamma(t) = (t, \sqrt{1 - t^{2}})\) after solving for \(y\). Although this works for the positive solutions to \(x^{2} + y^{2} = 1\), this does not cover the negative part of the circle.

\end{remark}

\end{example}


\begin{example}
		Consider the parametrized curve

\[
\gamma(t) = (\cos^{3}(t), \sin^{3}(t)), \;\; t \in \mathbb{R}
\]
		Since for all values of \(t\),

\[
\cos^{2}(t) + \sin^{2}(t) = 1
\]
, the coordinates \(x = \cos^{3}(t), y = \sin^{3}(t)\) of \(\gamma(t)\) satisfy

\[
x^{\frac{2}{3}} + y^{\frac{2}{3}} = 1
\]
		(since \(x^{\frac{2}{3}} + y^{\frac{2}{3}} = \cos^{3}(t)^{\frac{2}{3}} + \sin^{3}(t)^{\frac{2}{3}} = \cos^{2}(t) + \sin^{2}(t) = 1\). Thus, \(x^{\frac{2}{3}} + y^{\frac{2}{3}} = 1\) is a level curve that coincides with the image of \(\gamma\)

\end{example}


\begin{definition}
		Let \(f\) be a function, \(f\) is called a \textit{smooth function}\index{Smooth function} up to order \(n\) if the derivatives of \(f\) up to order \(n\) are continuous. For the purposes of this material, all functions will be assumed to be smooth for all \(n\)

\end{definition}


\begin{definition}
		Let \(\gamma\) be a parametrized curve. The derivative of order 1, \(\gamma'(t)\), is called the \textit{tangent vector}\index{Tangent vector} of \(\gamma\)

\end{definition}


\begin{theorem}
		Let \(\gamma\) be a parametrized curve. If the tangent vector of \(\gamma\) is constant, then the image of the curve is a straight line.

\begin{proof}
			Let \(\gamma\) be a parametrized curve with a constant tangent vector. That is, \(\forall t, \gamma'(t) = \vec{a}\) where \(\vec{a}\) is a constant vector. Then by component-wise integration, we have

\[
\gamma(t) = \int\gamma'(t) \dif t = \int\od{\gamma}{t} \dif t = \vec{a} \dif t = t\vec{a} + \vec{b}
\]
			where \(\vec{b}\) is a constant vector. This corresponds to the parametric equation for a straight line passing through point \(\vec{b}\).

\end{proof}

\end{theorem}


\begin{problem}
		Is \(\gamma(t) = (t^{2}, t^{4})\) a parametrization of \(y = x^{2}\)?

\end{problem}


\begin{solution}
		\(\gamma(t) = (t^{2}, t^{4})\) is only a valid parametrization of the positive \(x\) side of \(y = x^{2}\) since

\[
\gamma(t) = (t^{2}, t^{4}) \implies x = t^{2}, y = t^{4}
\]
		and

\[
y = t^{4} = (t^{2})^{2} = x^{2}
\]
		which satisfies the equation.

\end{solution}


\begin{problem}
		Find a parametrization of the level curve \(y^{2} - x^{2} = 1\).

\end{problem}


\begin{solution}
		We need to find a

\[
\gamma \colon \intoo{\alpha,\beta} \rightarrow \mathbb{R}^{2}
\]
		such that \(\gamma(t)\) satisfies the equation for every \(t \in \intoo{\alpha, \beta}\). First try \(x = t\), we get

\[
y^{2} - x^{2} = 1 \implies x = \pm\sqrt{-1 + y^{2}}
\]
		which is inconvenient for parametrizing. Now observe that

\[
\sec^{2}(t) - \tan^{2}(t) = 1
\]
. So let

\[
\gamma(t) = (\sec(t), \tan(t))
\]
. The possible values of \(t\) is then restricted to

\[
t \in (\frac{-\pi}{2}, \frac{\pi}{2}) \cup (\frac{\pi}{2}, \frac{3\pi}{2})
\]
.

\end{solution}


\begin{problem}
		Find the Cartesian equation of the parametrization

\[
\gamma(t) = (\cos^{2}(t), \sin^{2}(t)
\]
		We can rewrite the function \(\gamma(t)\) as

\[
\sin^{2}(t) = a\cos^{2}(t) + \phi(t)
\]

\[
= y = ax + \phi(t)
\]
		Observe that \(\sin^{2} + \cos^{2} = 1 \implies \sin^{2} = 1 - \cos^{2} = -\cos^{2} + 1\). Thus \(a = -1, \phi(t) = 1\) and we get the equation \(y = -x + 1\)
    \end{problem}

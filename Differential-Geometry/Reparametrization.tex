\subsection{Reparametrization}
	This section seeks to expand on the relationship between different possible parametrizations of a level curve.


\begin{definition}
		A parametrized curve \(\tilde{\gamma} \colon \intoo{\tilde{\alpha}, \tilde{\beta}} \rightarrow \mathbb{R}^{n}\) is a \textit{reparametrization}\index{Reparametrization} of a parametrized curve \(\gamma \colon \intoo{\alpha, \beta} \rightarrow \mathbb{R}^{n}\) if there is a smooth bijective map

\[
\phi : \intoo{\tilde{\alpha}, \tilde{\beta}} \rightarrow \intoo{\alpha, \beta}
\]
		known as the \textit{reparametrization map}\index{Reparametrization map} such that the inverse map

\[
\phi^{-1} \colon \intoo{\alpha, \beta} \rightarrow \intoo{\tilde{\alpha}, \tilde{\beta}}
\]
		is also a smooth function and

\[
\tilde{\gamma}(\tilde{t}) = \gamma(\phi(\tilde{t}) \; \forall \tilde{t} \in \intoo{\tilde{\alpha}, \tilde{\beta}}
\]

\end{definition}


\begin{example}
		We previously found that the circle \(x^{2} + y^{2} = 1\) has a parametrization

\[
\gamma(t) = (\cos(t), \sin(t))
\]
		Another valid parametrization of the circle is

\[
\tilde{\gamma}(t) = (\sin(t), \cos(t))
\]
		In order to show that \(\tilde{\gamma}\) is a valid reparametrizaiton of \(\gamma\), we must find a reparemtrization map such that

\[
(\cos(\phi(t)), \sin(\phi(t))) = (\sin(t), \cos(t))
\]
		Let \(\phi(t) = \frac{\pi}{2} - t\). Then we get the following

\[
(\cos(\phi(t)) = \cos(\frac{\pi}{2} - t) = \sin(t), \; \sin(\phi(t)) = \sin(\frac{\pi}{2} - t) = \cos(t)
\]

\end{example}


\begin{definition}
		A point \(\gamma(t)\) of a parametrized curve \(\gamma\) is called a \textit{regular point}\index{Regular point} if

\[
\gamma'(t) \not = \vec{0}
\]
		Otherwise (if \(\gamma(t)\) is not a regular point), we call \(\gamma(t)\) a \textit{singular point}\index{Singular point} of \(\gamma\).
		A curve is regular if all of the points on the curve are regular.

\end{definition}


\begin{proposition}
		Any reparametrization of a regular curve is regular


\begin{proof}
			Suppose \(\gamma\) is a regular curve and \(\tilde{\gamma}\) is a reparametrization of \(\gamma\). Let

\[
t = \phi(\tilde{t}) \text{and} \psi = \phi^{-1}
\]
			Now, differentiate both sides of \(\phi(\psi(t)) = t\) results in the following

\[
\od{\phi}{\tilde{t}} \od{\psi}{t} = 1
\]
			Now, since \(\tilde{\gamma}(\tilde{t}) = \gamma(\phi(\tilde{t})\), we can differentiate this to obtain

\[
\od{\tilde{\gamma}}{\tilde{t}} = \od{\gamma}{t} \od{\phi}{\tilde{t}}
\]
			From previous two differentiated equations, we can conclude that

\[
\od{\tilde{\gamma}}{\tilde{t}} \not = 0, \; \od{\phi}{\tilde{t}} \not = 0
\]
			This satisfies the definition of a regular curve.

\end{proof}

\end{proposition}


\begin{proposition}
		If \(\gamma(t)\) is a regular curve, its arc-length \(s\) starting at any point of \(\gamma\) is a smooth function of \(t\).

\begin{proof}
			Recall that by previous result, \(s\) is a differentiable function of \(t\) and

\[
\od{s}{t} = \norm{\gamma'(t)}
\]
			For simplicity, we will assume that \(\gamma\) is a plane curve of the form

\[
\gamma(t) = (u(t), v(t))
\]
			Where \(u\) and \(v\) are smooth functions of \(t\). Then we have

\[
\od{s}{t} = \sqrt{u'^{2} + v^{2}}
\]
			Now, observe that \(f(x) = \sqrt{x}\) is a smooth function on \(\intoo{0, \infty}\). Furthermore, by induction,

\[
\od[n]{f}{x} = (-1)^{n-1} \frac{(1)3(5)...(2n - 1)}{2^{n}} x^{\frac{-(2n + 1)}{2}}
\]
			Furthermore, since \(u\) and \(v\) are smooth on \(\intoo{0, \infty}\) so are \(u'\) and \(v'\) and an elementary composition of smooth functions is smooth so the function

\[
\sqrt{u'^{2} + v'^{2}}
\]
			is a smooth function. Then, we have enough information to conclude that

\[
\od{s}{t} = f(u'^{2} + v'^{2})
\]
			is a smooth function of \(t\) so \(s\) is smooth.

\end{proof}

\end{proposition}


\begin{proposition}
		A parametrized curve has a unit-speed reparametrization if and only if it is a regular curve.
    \end{proposition}

    \begin{corollary}
      Let \(\gamma\) be a regular curve and \(\tilde{\gamma}\) a unit-speed reparametrization
      of \(\gamma\)
      \[
\tilde{\gamma}(u(t)) = \gamma(t)
\]
      where \(u\) is a smooth function of \(t\). Then, if \(s\) is the arc-length of
      \(\gamma\), we have
      \[
u = \pm s + c
\]
      where c is a constant
    \end{corollary}

    \begin{example}
      This example shows that although a unit-speed reparametrization is always
      possible, it is not always convenient. Consider the logarithmic spiral
      \[
\gamma(t) = (e^{kt}\cos(t), e^{kt}\sin(t))
\]
      Seeking a unit-speed paremtrization, we have
      \[
\gamma'(t) = (-e^{kt}\sin(t) + ke^{kt}\cos(t), e^{kt}\cos(t) + ke^{kt}\sin(t))
\]
      and
      \[
\norm{\gamma'(t)}^{2} = e^{2kt}\sin^{2}(t) + k^{2}e^{2kt}\cos^{2}(t) +
        e^{2kt}\cos^{2}(t) + k^{2}e^{2kt}\sin^{2}(t) = (k^{2} + 1)e^{2kt}
\]
      so the arc-length starting from 0 is given by
      \[
s = \int^{t}_{0} \sqrt{k^{2}+1}e^{ku}\dif u = \frac{1}{k}\sqrt{k^{2} + 1}e^{kt}-1
\]
      then
      \[
t(s) = \frac{1}{k}\ln(\frac{ks}{\sqrt{k^{2}+1}}+1)
\]
      which gives us the ugly unit-speed reparemtrization of
      \[
\tilde{\gamma(s)} =
        ((\frac{ks}{\sqrt{k^{2}+1}}+1)\cos(\frac{1}{k}\ln(\frac{ks}{\sqrt{k^{2}+1}}+1)),
        (\frac{ks}{\sqrt{k^{2}+1}}+1)\sin(\frac{1}{k}\ln(\frac{ks}{\sqrt{k^{2}+1}}+1)))
\]
    \end{example}




\subsection{Space Curves}
Curves in \(\mathbb{R}^{3}\) or space curves are the main focus of study in this look at
differential geometry. We will show that the previously introduced idea of
curvature, together with a new concept, torsion, can determine any space curve up
to a direct isometry.

Recall that a tangent vector can be expressed as the first derivative of a
paremtrized curve. Then a unit-tangent vector is simply the tangent vector whose
magnitude is 1. Using this idea, we have a new definition that will be helpful
in introducing the concept of torsion.

\begin{definition}
  Let \(\gamma(s)\) be a unit-speed curve in \(\mathbb{R}^{3}\), and let \(\vec{t} = \gamma'\) be the
  unit-tangent vector of \(\gamma\). If the curvature is non-zero, we define the
  \textit{principal normal}\index{Principal normal} of \(\gamma\) at the point \(\gamma(s)\)
  to be the vector
  \[
\vec{n}(s) = \frac{1}{\kappa(s)}\vec{t}'(s)
\]
\end{definition}

Observe that since \(\norm{\vec{t}'} = \kappa\), \(\vec{n}\) is a unit vector.
Furthermore, \(\vec{t} \cdot \vec{t}' = 0\) so \(\vec{n}\) and \(\vec{t}\) are
perpendicular. This leads to our next definition.

\begin{definition}
  Let \(\vec{t}\) be a unit-tangent vector of a paremtrized curve \(\gamma\) and let
  \(\vec{n}\) be its principal norm. Then the vector
  \[
\vec{b}(s) = \vec{t}(s) \times \vec{n}(s)
\]
  is called the \textit{binormal vector}\index{Binormal vector} of \(\gamma\) at the
  point \(\gamma(s)\).
\end{definition}

The set of the unit-tangent vector, the principal norm and the binormal vector
\(\{\vec{t}, \vec{n}, \vec{b}\}\) form an orthonormal basis of \(\mathbb{R}^{3}\). If we
apply the vector product rule to the binormal norm, we obtain the equation
\[
\vec{b}' = \vec{t}'\times \vec{n} + \vec{t} \times \vec{n}' = \vec{t} \times \vec{n}'
\]
Furthermore, by definition,
\[
\vec{n}(s) = \frac{1}{\kappa(s)}\vec{t}'(s) \implies \vec{t}'\times \vec{n} = \kappa
  \vec{n}\times \vec{n} = 0
\]
so \(\vec{b}'\) is perpendicular to both \(\vec{b}\) and \(\vec{t}\). Thus it is
parallel to \(n\). This leads us to our next definition

\begin{definition}
  Let \(\vec{b}\) be a binormal vector of a parametrized curve \(\gamma\). The
  \textit{torsion}\index{torsion} of \(\gamma\) at a point \(\gamma(s)\) is given by \(\tau\) where
  \[
\vec{b}' = -\tau \vec{n}
\]
\end{definition}

Similar to curvature, it is possible to express the torsion of a curve in terms
of \(\gamma\) itself rather than a unit-speed reparametrization.

\begin{proposition}
  Let \(\gamma(t)\) be a regular curve in \(\mathbb{R}^{3}\) with non-zero curvature. Then the
  torsion of \(\gamma\) is given by
  \[
\tau = \frac{(\gamma' \times \gamma'')\cdot \gamma'''}{\norm{\gamma' \times \gamma''}^{2}}
\]

  \begin{proof}
    Similar to the proof for curvature, this proof is omitted due to lengthy computations.
  \end{proof}
\end{proposition}

\begin{proposition}
  Let \(\gamma\) be a regular curve in \(\mathbb{R}^{3}\) with nowhere vanishing curvature. Then
  the image of \(\gamma\) is contained in a plane if and only if \(\tau\) is zero at every
  point of the curve

  \begin{proof}
    Let \(\gamma\) be unit-speed (reparametrizing does not change torsion). Let \(s\) be
    the parameter of \(\gamma\). Next, suppose the image of \(\gamma\) is contained by the
    plane \(\vec{n}\cdot \vec{N} = d\) where \(\vec{N}\) is a constant unit vector and
    \(\vec{v}\in \mathbb{R}^{3}\). Differentiating \(\gamma\cdot \vec{N} = d\) with respect to \(s\), we have
    \[
\vec{t}\cdot \vec{N} = 0 \implies \vec{t}' \cdot \vec{N} = 0 \implies \kappa \vec{n}\cdot \vec{N}
      = 0
\]
    Since \(\vec{t}\) and \(\vec{n}\) are perpendicular to \(\vec{N}\) so \(\vec{b}\) is
    parallel to \(\vec{N}\) and is a constant vector. So \(\vec{b}' = 0\) so \(\tau =
    0\).
    The converse is straightforward.
  \end{proof}
\end{proposition}

\begin{theorem}
  Let \(\gamma\) be a unit-speed curve in \(\mathbb{R}^{3}\) with non-zero curvature. Then the
  following relations hold
  \[
\vec{t}' = \kappa \vec{n}
\]
  \[
\vec{n}' = -\kappa \vec{t} + \tau \vec{b}
\]
  \[
\vec{b} = -\tau \vec{n}
\]
  These equations are called the \textit{Frenet-Serret
    equations}\index{Frenet-Serret equations}
\end{theorem}

The following result is an immediate consequence of the Frenet-Serret equations

\begin{proposition}
  Let \(\gamma\) be a unit-speed curve in \(\mathbb{R}^{3}\) with constant curvature and zero
  torsion. Then \(\gamma\) is a parametrizaiton of a circle.
  \begin{proof}
    This immediately follows from the previous proposition which states that any
    curve with zero torsion contains the image of \(\gamma\) (since a circle has a
    constant curvature). However, a more instructive proof follows. We
    previously showed that the binormal vector of a zero torsion curve is
    constant and \(\gamma\) is contained in a plane perpendicular to \(\vec{b}\). Now
    \[
\od{}{s}(\gamma + \frac{1}{\kappa}\vec{n} = \vec{t} + \frac{1}{\kappa}\vec{n}' = 0)
\]
    From the Frenet-Serret equation, and the fact that curvature is constant, we
    have
    \[
\vec{n}' = -\kappa \vec{t} + \tau \vec{b} = -\kappa \vec{t}
\]
    Then \(\gamma + \frac{1}{\kappa}\vec{n}\) is a constant vector and
    \[
\norm{\gamma - \gamma + \frac{1}{\kappa}\vec{n}} = \frac{1}{k}
\]
    So \(\gamma\) lies on a sphere \(S\) with center \(\gamma + \frac{1}{\kappa}\vec{n}\) and radius
    \(\frac{1}{k}\).
  \end{proof}
\end{proposition}

\begin{theorem}
  Let \(\gamma(s)\) and \(\tilde{\gamma}(s)\) be two unit-speed curves in \(\mathbb{R}^{3}\) with the
  same \(\kappa(s)>0\) and \(\tau\) for all \(s\). Then there is a direct isometry \(M\) of
  \(\mathbb{R}^{3}\) such that
  \[
\tilde{\gamma}(s) = M(\gamma(s))
\]
\end{theorem}

\subsection{Arc-length}
    Recall that for a vector \(v\) in \(\mathbb{R}^{n}\), its magnitude or length is given by
    \[
\norm{v} = \sqrt{v_{1}^{2} + \ldots + v_{n}^{2}}
\]
    If \(u\) is another vector in \(\mathbb{R}^{n}\), the length of the line joining \(u\) and
    \(v\) is given by \(\norm{u-v}\). So for a parametrized curve \(\gamma\), if we take \(
    \delta t\) to be very small, we get a (nearly) straight line between the points \(\gamma
    (t + \delta t)\) and \(\gamma(t)\). In other words, for very small \(\delta\),
        \[
\gamma(t + \delta t) - \gamma(t) = \norm{\gamma(t + \delta t) - \gamma(t)}
\]
    Furthermore, taking \(\delta\) to be very small, we get
        \[
\frac{\gamma(t + \delta t) - \gamma(t)}{\delta t} = \gamma'(t)
\]
    so the length between the two points is
        \[
\gamma(t + \delta t) - \gamma(t) = \norm{\gamma(t + \delta t) - \gamma(t)}
\]
        \[
= \norm{\gamma'(t)}\delta t
\]
    Integrating over this yields the length of the curve (arclength) for the integration bounds. This idea is defined formally below.

    \begin{definition}
        The \textit{arc-length}\index{Arc-length} of a curve \(\gamma\) starting at the point \(\gamma(t_{0} \) is given by the function \(s(t) \) defined by
        \[
s(t) = \int_{t_{0}}^{t} \norm{g'(u)} \dif u
\]
    \end{definition}

    \begin{example}
        Consider the logarithmic spiral given by
            \begin{figure}[h]
                \centering
                \caption*{\(\gamma(t) = (e^{kt}\cos(t), e^{kt}\sin(t))\)}
                \begin{tikzpicture}[scale = 0.125, line cap = round,line join = round,> = triangle 45,x = 1cm,y = 1cm]
                    \begin{axis}
[
                        x = 1cm,y = 1cm,
                        axis lines = middle,
                        xmin = -12,
                        xmax = 12,
                        ymin = -12,
                        ymax = 12,
                        xtick = \empty,
                        ytick = \empty]
                        \clip(-12,-12) rectangle (12,12);
                        \draw[line width = 1.2pt,
                        smooth,
                        samples = 100,
                        domain = 0:25.132741228718345] plot[parametric] function{2.718281828459045**(0.1*t)*cos((t)),2.718281828459045**(0.1*t)*sin((t))};
                    \end{axis}
                \end{tikzpicture}
            \end{figure}
        Where \(k\) is a non-zero constant. We have
            \[
\gamma' = (e^{kt}(k \cos(t) - \sin(t)), e^{kt}(k \sin(t) + \cos(t))
\]
        and from this we get
            \[
\norm{\gamma}^{2} = e^{2kt}(k \cos(t) - \sin(t))^{2} + e^{2kt}(k \sin(t) \cos(t))^{2} = (k^{2} + 1)e^{2kt}
\]
        We square the magnitude for simplicity of integration. Taking the starting point
            \[
\gamma(0) = (e^{0} \cos(0), e^{0} \sin(0)) = (1, 0)
\]
        for example, we get the arc-length of \(\gamma\) from \(0\) to \(t\) given by
            \[
\int_{0}^{t} \sqrt{k^{2} + 1}e^{ku} \dif u = \frac{\sqrt{k^{2} + 1}}{k}(e^{kt} -1)
\]
    \end{example}

    \begin{definition}
        If \(\gamma : (\alpha, \beta) \rightarrow \mathbb{R}^{n}\) is a parametrized curve, the \textit{speed}\index{Speed} at the point \(\gamma(t)\) is given by \(\norm{\gamma'(t)}\). Additionally, \(\gamma\) is said to be a \textit{unit-speed curve}\index{Unit-speed curve} if \(\gamma'(t)\) is a unit vector for all \(t \in (\alpha, \beta)\)
    \end{definition}

    Recall the dot product between two vectors
        \[
\vec{a} = (a_{1}, a_{2}, \ldots, a_{n}) \text{and} \vec{b} = (b_{1}, b_{2}, \ldots, b_{n})
\]
    where \(a, b \in \mathbb{R}^{n}\) is given by
        \[
\vec{a} \cdot \vec{b} = \sum_{i = 1}^{n} a_{i}b_{i} = a_{1}b_{1} + a_{2}b_{2} + \ldots + a_{n}b_{n}
\]
    Now, if \(a,b\) are smooth and functions of \(t\), we can apply the"product rule" from calculus to obtain
        \[
\od{}{t}(\vec{a} \cdot \vec{b}) = \od{\vec{a}}{t} \cdot \vec{b} + \vec{a} \cdot \od{\vec{b}}{t}
\]

    \begin{proposition}
        Let \(\vec{n}(t)\) be a unit vector that is a smooth function of parameter \(t\). Then the following dot product satisfies
            \[
\vec{n}'(t) \cdot \vec{n}(t) = 0
\]
        for all \(t\). In other words, \(\vec{n}'(t)\) is perpendicular to \(\vec{n}(t)\) for all \(t\).
        \begin{proof}
            By the definition of unit vector,
                \[
\vec{n} \cdot \vec{n} = 1
\]
            Differentiating both sides with respect to \(t\) gives
                \[
\vec{n}' \cdot \vec{n} + \vec{n} \cdot \vec{n}' = 0
\]
            so
                \[
2 \vec{n}' \cdot \vec{n} = 0
\]
. Take \(\vec{n} = \gamma'\) to satisfy the desired equality.
        \end{proof}
    \end{proposition}

   \begin{problem}
        Calculate the arc-length of the catenary
        \begin{figure}[h]
            \centering
            \caption*{\(\gamma(t) = t, \cosh(t)\)}
            \begin{tikzpicture}[scale = 0.25, line cap = round,line join = round,> = triangle 45,x = 1cm,y = 1cm]
                \begin{axis}
[
                    x = 1cm,y = 1cm,
                    axis lines = middle,
                    xmin = -12,
                    xmax = 12,
                    ymin = -12,
                    ymax = 12,
                    xtick = \empty,
                    ytick = \empty]
                \clip(-12,-12) rectangle (12,12);
                \draw[line width = 2pt, smooth,samples = 100,domain = 0:6.283185307179586] plot[parametric] function{t,cosh(t)};
                \end{axis}
            \end{tikzpicture}
        \end{figure}
    \end{problem}

    \begin{solution}
        We begin by finding the first derivative and its length
            \[
\gamma'(t) = (1, \sinh(t))
\]
        and the length
            \[
\norm{\gamma'(t)} = \sqrt{1 + \sinh^{2}(t)} = \sqrt{\cosh^{2}(t) = \cosh(t)}
\]
        The arc-length starting at the point \(\gamma(0) = (0, \cosh(0)) = (0, 1)\) is then given by
            \[
s = \int_{0}^{t} \cosh(u) \dif u = \eval{\sinh(u)}_{0}^{t} = \sinh(t)
\]
    \end{solution}

    \begin{problem}
        Show that the following curve is unit-speed
            \[
\gamma(t) = (\frac{1}{3} (1 + t)^{\frac{3}{2}}, \frac{1}{3} (1 - t)^{\frac{3}{2}}, \frac{t}{\sqrt{2}})
\]
    \end{problem}

    \begin{solution}
A unit vector is a vector whose magnitude is 1. We must show that \(\gamma'(t)\) is a unit vector for all \(t \in \intoo{-\infty, \infty}\) (since the domain of \(t\) is all real numbers). First we find the first derivative of \(\gamma\) with respect to \(t\)
            \[
\gamma'(t) = (\frac{\sqrt{t + 1}}{2}, -\frac{\sqrt{t - 1}}{2}, \frac{1}{2})
\]
        The magnitude of this vector (that is a smooth function of \(t\)) is given by
            \[
\norm{\gamma'(t)}^{2} = \frac{\sqrt{t + 1}}{2}^{2} + \frac{\sqrt{t - 1}}{2}^{2} + \frac{1}{2}^{2} = \frac{t + 1}{4} + \frac{-t + 1}{4} + \frac{1}{2} = 1
\]
        Thus \(\gamma(t)\) is a unit vector
      \end{solution}

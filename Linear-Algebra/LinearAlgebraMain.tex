%%% Tex-command-extra-options: "-shell-escape"
%%% TeX-master: t
%%% End:


\documentclass[11pt]{article}
\usepackage[T1]{fontenc}
\usepackage{amsmath, amsrefs, amssymb, amsthm, amsxtra, commath, fancyhdr, mathrsfs, minted, systeme, wasysym}
\usepackage{caption, pgfplots}
\usepackage{setspace}
\onehalfspacing
\pgfplotsset{compat = 1.15}
\usepackage{imakeidx}
\usepackage{hyperref}
\usetikzlibrary{arrows}
\hypersetup{pdftitle = {Numerical Methods Notes}, pdfpagemode=FullScreen, linktoc=all, colorlinks=true}
\theoremstyle{definition}
\newtheorem{definition}{Definition}[section]
\newtheorem{example}{Example}[section]
\newtheorem{problem}{Problem}[section]
\newtheorem{solution}{Solution}[section]
\theoremstyle{plain}
\newtheorem{theorem}{Theorem}[section]
\newtheorem{proposition}[definition]{Proposition}
\newtheorem{corollary}{Corollary}[definition]
\newtheorem{lemma}{Lemma}[theorem]
\theoremstyle{remark}
\newtheorem*{remark}{Remark}
\makeindex[columns = 2, title = Definitions, intoc, columnseprule]
\renewcommand{\subsubsectionmark}[1]{\markright{\thesubsubsection\ #1}}

\pagestyle{fancy}
\title{Linear Algebra Notes}
\fancyhf{}
\fancyhead{}
\fancyhead[L]{\scshape\leftmark}
\let\oldsubsection\subsection
\makeatletter
\def\subsection{%
\@ifstar{\@Starred}{\@nonStarred}%
}
\def\@Starred{%
\@ifnextchar[%
{\GenericWarning{}{Warning: A starred section can not have parameters. I am going to ignore them!}\@StarredWith}%
{\@StarredWithout}%
}      
\def\@StarredWith[#1]#2{%
\oldsubsection*{#2}%
\renewcommand\leftmark{#1}%
}
\def\@StarredWithout#1{
\oldsubsection*{#1}%
\renewcommand\leftmark{#1}
}
\def\@nonStarred{%
\@ifnextchar[%
{\@nonStarredWith}%
{\@nonStarredWithout}%
}
\def\@nonStarredWith[#1]#2{%
\oldsubsection[#1]{#2}%
\renewcommand\leftmark{#1}%
}
\def\@nonStarredWithout#1{%
\oldsubsection{#1}%
\renewcommand\leftmark{#1}%
}
\makeatother
\fancyhead[R]{\thepage}
\date{}
\author{Hunter Smith}

\begin{document}

\maketitle
\tableofcontents

\begin{document}
  \maketitle
  \tableofcontents
  \newpage

  \section{Vector Spaces}
      
\subsection{Vector Spaces}

\subsubsection*{}

\begin{definition}
		A \textit{vector space}\index{Vector Space} \(V\) over a field \(F\) consists of a set on which two operations are defined so that for each \(x,y\in V\) there exists \(x+y\in V\) and \(ax\in V\;\forall a\in F\) such that the following hold:

\[
\forall x, y \in V, \; x + y = y + x
\]

\[
\forall x, y, z \in V, \; (x + y) + z = x + (y + z)
\]

\[
\exists 0 \in V | x + 0 = x, \; \forall x \in V
\]

\[
\forall x \in V, \; \exists y \in V | x + y = 0
\]

\[
\forall x \in V, \; 1x = x
\]

\[
\forall a, b \in F, x \in V, \; (ab)x = a(bx)
\]

\[
\forall a \in F \text{and} x, y \in V, \; a(x + y) = ax + ay
\]

\[
\forall a,b\in F,x\in V,\; (a+b)x = ax+bx
\]

\end{definition}


\begin{example}
		The set of \(n\)-tuples from a field \(F\) is denoted \(F^n\). This set is a vector space over \(F\) with the operations of vector addition and scalar multiplication. That is, if \(u = (a_{1},a_{2},\dots,a_{n}), v = (b_{1},b_{2},\dots,b_{n})\) and \(c\in F\)\\
then \(u + v = (a_{1}+b_{1},a_{2}+b_{2},\dots,a_{n}+b_{n})\) and \(cu = (ca_{1},ca_{2},\dots,ca_{n})\).

\end{example}


\begin{example}
		The set of \(m\times n\) matrices with entries from a field \(F\) is a vector space, denoted \(M_{m\times n}(F)\), with the operations of matrix addition and scalar multiplication. Let \(A,B\in M_{m\times n}(F)\) and let \(c\in F\). Then

\[
(A + B)_{ij} = A_{ij} + B_{ij}, c(A_{ij}) = cA_{ij}
\]

\end{example}


\begin{example}
		Let

\[
f(x) = a_{n}x^{n} + a_{n-1}x^{n-1} +\dots+ a_{1}x + a_{0} \text{and} g(x) = b_{n}x^{n} + b_{n-1}x^{n-1} +\dots+ b_{1}x + b_{0}
\]
		be polynomials with coefficients from a field \(F\). Define

\[
f(x)+g(x) = (a_{n}+b_{n})x^{n}+(a_{n-1}+b_{n-1})x^{n-1}+\dots+(a_{1}+b_{1})x+(a_{0}+b_{0})
\]

\[
\text{and}
\]

\[
cf(x) = ca_{n}x^{n}+ca_{n-1}x^{n-1}+\dots+ca_{1}x+ca_{0}
\]
		 Under this construction, the set of polynomials with coefficients from a field \(F\) is a vector space and is denoted \(P(F)\)

\end{example}


\begin{example}
		Let \(S = \{(a_{1}, a_{2})|a_{1}, a_{2}\} \in \mathbb{R}\) and define the following two\\
 operations:

\[
(a_{1}, a_{2}) + (b_{1}, b_{2}) = (a_{1} + b_{1}, a_{2} - b_{2})
\]

\[
c(a_{1}, a_{2}) = (ca_{1}, ca_{2})
\]
		Notice that conditions 1,2 and 8 of a vector space fail to hold so S is not a vector space.

\end{example}


\begin{theorem}[Cancellation Law of Vector Addition]
		if \(x,y,z\in V\) and \(x+z = y+z\), then \(x = y\)

\begin{proof}
			By condition 4 of a vector space, there is a vector \(v\) such that \(z+v = 0\). Now,

\[
x = x + 0 = x + (z + v) = (x + z) + v = (y + z) + v = y + (z + v) = y + 0 = y
\]
			by conditions 2 and 3 of a vector space

\end{proof}

\end{theorem}


\begin{theorem}[Zero Vector Properties]
		In any vector space \(V\), the following hold:

\[
\ forall x \in V, \; 0x = 0
\]

\[
\ forall a\in F, x \in V, \; (-a)x = -a(x) = a(-x)
\]

\[
\ forall a\in F, \; a0 = 0
\]

\end{theorem}

\subsubsection*{Problems}


\begin{problem}
		Let \(S = \{0,1\}\) and \(F = R\). Show that \(f = g\) and \(f+g = h\) where \(f(x) = st+1\), \(g(x) = 1 + 4t + 2t^{2}\), \(h(x) = 5^{t} + 1\).

\end{problem}


\begin{solution}
		Since the set \(S\) consists of only 0 and 1, the following is sufficient to show that \(f = g\) and \(f+g = h\):

\[
f(0) = 2(0) + 1 = 1 = 1 + 4(0) + 2(0)^{2} = g(0)
\]

\[
f(1) = 2(1) + 1 = 3 = 1 + 4(1) + 2(1)^{2} = g(1)
\]

\[
f(0) + g(0) = 1 + 1 = 5^{0}+1 = h(0)
\]

\[
f(1) + g(1) = 3 + 3 = 6 = 5^{1}+1 = h(1)
\]

\end{solution}


\begin{problem}
		Show that

\[
(a + b)(x + y) = ax + ay + bx + by)
\]
		for any \(x, y \in V\) and \(a, b \in F\)

\end{problem}


\begin{solution}

\begin{proof}
			Let \(V\) be a vector space with \(x, y \in V\). Then by condition 7 and 8 of a vector space,

\[
(a + b)(x + y) = a(x + y) + b(x + y) = ax + ay + bx + by
\]

\end{proof}
\end{solution}
      
\subsection{Subspaces}

\subsubsection*{}

\begin{definition}
		A subset \(W\) of a vector space \(V\) over a field \(F\) is called a \textit{subspace}\index{subspace} of \(V\) if \(W\) is a vector space over \(F\) with the operations of addition and scalar multiplication defined on \(V\).

\end{definition}


\begin{theorem}[Subspace of a Vector Space]
		Let \(V\) be a vector space and \(W \subseteq V\). Then \(W\) is a subspace of \(V\) if and only if the following conditions hold:

\[
0 \in V
\]

\[
\forall x, y \in W, \; x + y \in W
\]

\[
\forall c \in F, x \in W, \; cx \in W
\]

\end{theorem}


\begin{example}
		The set \(W\) of all symmetric matrices (matrices such that \(A^{t} = A\)) is a subspace of \(M_{m\times n}(F)\). Since the zero matrix has all zero entries, clearly the zero matrix is in \(W\). If \(A,B\in W\), then \((A+B)^{t} = A^{t}+B^{t} = A+B\) so \(W\) is closed under addition. If \(A\in W\) and \(c\in F\), then \(cA^{t} = c(A^{t}) = c(A) = cA\) so \(W\) is closed under scalar multiplication.

\end{example}


\begin{example}
		The set \(W\) of diagonal matrices is a subspace of \(M_{n \times n}(F)\). Clearly, the zero matrix is diagonal so it is in \(W\). Further, if \(A,B\in W\) then \((A+B)_{ij} = A_{ij}+B_{ij}\) and when \(i\not = j\), \(A_{ij} = B_{ij} = 0\). So \(A_{ij}+B_{ij}\) is diagonal and in \(W\). The same argument applies to closure under scalar multiplication.

\end{example}


\begin{theorem}[Intersection of Subspaces]
		Any intersection of subspaces of a vector space \(V\) is a subspace.

\begin{proof}
			Since every subspace contains the zero vector, the intersection of subspaces contains the zero vector. The closure of addition and multiplication follow from the definition of closure.

\end{proof}

\end{theorem}

\subsubsection*{Problems}


\begin{problem}
		Prove that \((A^{t})^{t} = A\) for all \(A \in M_{m\times n}(F)\)

\end{problem}


\begin{solution}
		Let \(A \in M_{m\times n}(F)\). Then by the definition of transpose,

\[
(A^{t})^{t} = (A^{t})_{ij}^{t} = (A_{ji})^{t} = A_{ij} = A
\]

\end{solution}


\begin{problem}
		Prove that diagonal matrices are symmetric matrices

\end{problem}


\begin{solution}
		Let \(A\) be a diagonal matrix. Then where \(i \not = j\), \(A_{ij} = 0\). Additionally, for all \(i = j\), we have \(A_{ij} = A_{ji}\). Then

\[
A^{t} = (A^{t})_{ij} = A_{ji} = A_{ij} = A
\]
		and \(A\) is symmetric by definition.

\end{solution}


\begin{problem}
		Prove that the set \(W_{1} = \{(a_{1}, a_{2},\dots, a_{n} \in F^{n}|a_{1} + a_{2} +\dots+ a_{n} = 0\}\) is a subspace and the space \(W_{2} = \{(a_{1}, a_{2},\dots, a_{n} \in F^{n}|a_{1} + a_{2} +\dots+ a_{n} = 1\}\) is not.

\end{problem}


\begin{solution}
		Let \(W_{1}\) be defined as above. Since \(a_{1} + a_{2} +\dots+ a_{n} = 0\), the zero vector is in \(W_{1}\). Let \(\{a_{n}\}, \{b_{n}\} \in W_{1}\). Then \(\{a_{n}\} + \{b_{n}\} = (a_{1} + a_{2} +\dots+ a_{n}) + (b_{1} + b_{2} +\dots+ b_{n}) = 0 + 0 = 0 \in W_{1}\). For \(c\in F\), \(c\{a_{n}\} = c(a_{1} + a_{2} +\dots+a_{n}) = c(0) = 0 \in W_{1}\). So \(W_{1}\) is a subspace. Define \(W_{2}\) as above, the zero vector is not in \(W_{2}\) so it is not a subspace.

\end{solution}
      
\subsection{Linear Combinations and Systems of Equations}

\subsubsection*{}

\begin{definition}
		Let \(V\) be a vector space and \(S\) a nonempty subset of \(V\). A vector \(v \in V\) is called a \textit{linear combination}\index{Linear combination} of vectors of \(S\) if there exist a finite number of vectors \(u_{1},u_{2},\dots,u_{n}\in S\) and scalars \(a_{1},a_{2},\dots,a_{n}\in F\) such that \(v = a_{1}u_{1}+a_{2}u_{2}+\dots+a_{n}u_{n}\).

\end{definition}


\begin{example}
		The claim is that

\[
2x^{3} - 2x^{2} + 12x - 6
\]
		is a linear combination of

\[
x^{3} - 2x^{2} - 5x - 3 \text{and} 3x^{3} - 5x^{2} - 4x - 9
\]
		in \(P_{3}(R)\), but

\[
3x^{3} - 2x^{2} + 7x + 8
\]
		is not. For case 1, we need to find scalars \(a,b\) such that

\[
2x^{3} - 2x^{2} + 12x - 6 = a(x^{3} - 2x^{2} - 5x - 3) + b(3x^{3} - 5x^{2} - 4x - 9)
\]

\[
= (a + 3b)x^{3} + (-2a - 5b)x^{2} + (-5a - 4b)x + (-3a - 9b)
\]
		Which results in the following

\[
a + 3b = 2
\]

\[
-2a - 5b = -2
\]

\[
-5a - 4b = 12
\]

\[
-3a - 9b = -6
\]
		Now, we take a multiple of the first equation, and add it to the remaining equations to eliminate \(a\) (2, 5 and 3 respectively). This results in the following equations

\[
a + 3b = 2
\]

\[
2(a + 3b) + (-2a - 5b) = b = 2(2) - 2 = 2
\]

\[
5(a + 3b) + (-5a - 4b) = 11b = 5(2) + 12
\]

\[
3(a + 3b) + (-3a - 9b) = 0b = 3(2) - 6 = 0
\]

		Solving this yields \(a = -4, b = 2\) hence

\[
2x^{3} - 2x^{2} + 12x - 6 = -4(x^{3} - 2x^{2} - 5x - 3) + 2(3x^{3} - 5x^{2} - 4x - 9)
\]

		For the case involving

\[
3x^{3} - 2x^{2} + 7x + 8
\]
,we need to show that there are no scalars \(a,b\) such that

\[
3x^{3} - 2x^{2} + 7x + 8 = a(x^{3} - 2x^{2} - 5x - 3) + b(3x^{3} - 5x^{2} - 4x - 9)
\]

\[
= (a + 3b)x^{3} + (-2a - 5b)x^{2} + (-5a - 4b)x + (-3a - 9b)
\]
		Using the same technique as previous, we get the system of equations

\[
a + 3b = 3
\]

\[
-2a - 5b = -2
\]

\[
-5a - 4b = 7
\]

\[
-3a - 9b = 8
\]
		then

\[
a + 3b = 3
\]

\[
2(a + 3b) + (-2a - 5b) = b = 2(3) - 2 = 4
\]

\[
5(a+3b) + (-5a-4b) = 11b = 5(3)+7 = 22
\]

\[
3(a+3b) + (-3a-9b) = 0b = 0 = 3(3) + 8 = 17
\]
		The inconsistency \(0 = 17\) means that his equation has no solutions. So

\[
3x^{3}-2x^{2}+7x+8
\]
		is not a linear combination of the two equations.

\end{example}


\begin{definition}
		Let \(S\) be a nonempty subset of a vector space \(V\). The \textit{span}\index{Span} of \(S\) denoted by span\((S)\), is the set consisting of all linear combinations of the vectors in S.

\end{definition}

	Observe that in \(\mathbb{R}^{3}\), the spanning set of \(\{(1,0,0),(0,1,0)\}\) consists of all vectors in \(\mathbb{R}^{3}\) of the form \(a(1,0,0)+b(0,1,0)\) (by the definition of linear combinations) which is equivalent to \((a,b,0)\) or a point on the \(xy\) plane. Thus, this spanning set contains all points of the \(xy\) plane and is a subset of \(\mathbb{R}\).


\begin{theorem}[Span is a Subspace]
		The span of any subset \(S\) of a vector space \(V\) is a subspace of \(V\) that contains \(S\). Moreover, any subspace of \(V\) that contains \(S\) must also contain the span of \(S\).


\begin{proof}[Incomplete Proof]
			If \(S = \emptyset\), the proof is immediate since the span\((\emptyset) = \{0\}\) and \(\{0\}\) is a subspace of any subspace \(V\) (and in this case, \(\{0\}\) contains \(S\) since any set contains the empty set). Let \(S\) be a nonempty subset of \(V\). Then \(S\) contains a vector, call it \(z\). Then \(0z = 0\) so by the definition of span, \(0\in\) span\((S)\). Let \(x,y\in\) span\((S)\). Then there exists vectors \(u_{1},u_{2},\dots,u_{m}\), \(v_{1},v_{2},\dotsv_{n}\in S\) and scalars \(a_{1},a_{2},\dots,a_{m}\), \(b_{1},b_{2},\dotsb_{n}\in V\) such that

\[
x = a_{1}u_{1} + a_{2}u_{2} +\dots+ a_{m}u_{m} \text{and} y = b_{1}v_{1} + b_{2}v_{2} +\dots+ b_{n}v_{n}
\]
			Then

\[
x + y = a_{1}u_{1} + a_{2}u_{2} +\dots+ a_{m}u_{m} + b_{1}v_{1}+b_{2}v_{2} +\dots+ b_{n}v_{n}
\]
			is a linear combination of vectors in \(S\) so \(x + y \in\) span\((S)\). Similarly,

\[
cx = ca_{1}u_{1} + ca_{2}u_{2} +\dots+ ca_{m}u_{m}
\]
			which is also in span\((S)\). Thus span\((S)\) contains the zero vector, is closed under addition and scalar multiplication, so it is a subspace of V.

\end{proof}

\end{theorem}


\begin{definition}
		A subset \(S\) of a vector space \(V\) \textit{generates}\index{generates} or \textit{spans}\index{spanning set} \(V\) if span\((S) = V\).

\end{definition}


\begin{example}
		The vectors

\[
(1, 1, 0), (1, 0, 1) \text{and} (0, 1, 1)
\]
		span \(\mathbb{R}^{3}\) since any arbitrary vector in \(\mathbb{R}^{3}\) is a linear combination of these 3 vectors. It is possible to compute the scalars \(r,s,t\) which solve the equation

\[
r(1, 1, 0) + s(1, 0, 1) + t(0, 1, 1) = (a_{1}, a_{2}, a_{3})
\]
		by

\[
r = \frac{1}{2}(a_{1} + a_{2} - a_{3}), s = \frac{1}{2}(a_{1} - a_{2} + a_{3}), t = \frac{1}{2}(-a_{1} + a_{2} + a_{3})
\]

\end{example}


\begin{example}
		The polynomials

\[
x^{2} + 3x - 2, 2x^{2} + 5x - 3 \text{and} -x^{2} - 4x + 4
\]
		span or generate \(P_{2}(R)\) since each polynomial in \(P_{2}(R)\) is a linear combination of these 3 polynomials, and they are all in \(P_{2}(R)\).

\end{example}

  \printindex
\end{document}
\end{document}

\subsection{Linear Combinations and Systems of Equations}

\subsubsection*{}

\begin{definition}
		Let \(V\) be a vector space and \(S\) a nonempty subset of \(V\). A vector \(v \in V\) is called a \textit{linear combination}\index{Linear combination} of vectors of \(S\) if there exist a finite number of vectors \(u_{1},u_{2},\dots,u_{n}\in S\) and scalars \(a_{1},a_{2},\dots,a_{n}\in F\) such that \(v = a_{1}u_{1}+a_{2}u_{2}+\dots+a_{n}u_{n}\).

\end{definition}


\begin{example}
		The claim is that

\[
2x^{3} - 2x^{2} + 12x - 6
\]
		is a linear combination of

\[
x^{3} - 2x^{2} - 5x - 3 \text{and} 3x^{3} - 5x^{2} - 4x - 9
\]
		in \(P_{3}(R)\), but

\[
3x^{3} - 2x^{2} + 7x + 8
\]
		is not. For case 1, we need to find scalars \(a,b\) such that

\[
2x^{3} - 2x^{2} + 12x - 6 = a(x^{3} - 2x^{2} - 5x - 3) + b(3x^{3} - 5x^{2} - 4x - 9)
\]

\[
= (a + 3b)x^{3} + (-2a - 5b)x^{2} + (-5a - 4b)x + (-3a - 9b)
\]
		Which results in the following

\[
a + 3b = 2
\]

\[
-2a - 5b = -2
\]

\[
-5a - 4b = 12
\]

\[
-3a - 9b = -6
\]
		Now, we take a multiple of the first equation, and add it to the remaining equations to eliminate \(a\) (2, 5 and 3 respectively). This results in the following equations

\[
a + 3b = 2
\]

\[
2(a + 3b) + (-2a - 5b) = b = 2(2) - 2 = 2
\]

\[
5(a + 3b) + (-5a - 4b) = 11b = 5(2) + 12
\]

\[
3(a + 3b) + (-3a - 9b) = 0b = 3(2) - 6 = 0
\]

		Solving this yields \(a = -4, b = 2\) hence

\[
2x^{3} - 2x^{2} + 12x - 6 = -4(x^{3} - 2x^{2} - 5x - 3) + 2(3x^{3} - 5x^{2} - 4x - 9)
\]

		For the case involving

\[
3x^{3} - 2x^{2} + 7x + 8
\]
,we need to show that there are no scalars \(a,b\) such that

\[
3x^{3} - 2x^{2} + 7x + 8 = a(x^{3} - 2x^{2} - 5x - 3) + b(3x^{3} - 5x^{2} - 4x - 9)
\]

\[
= (a + 3b)x^{3} + (-2a - 5b)x^{2} + (-5a - 4b)x + (-3a - 9b)
\]
		Using the same technique as previous, we get the system of equations

\[
a + 3b = 3
\]

\[
-2a - 5b = -2
\]

\[
-5a - 4b = 7
\]

\[
-3a - 9b = 8
\]
		then

\[
a + 3b = 3
\]

\[
2(a + 3b) + (-2a - 5b) = b = 2(3) - 2 = 4
\]

\[
5(a+3b) + (-5a-4b) = 11b = 5(3)+7 = 22
\]

\[
3(a+3b) + (-3a-9b) = 0b = 0 = 3(3) + 8 = 17
\]
		The inconsistency \(0 = 17\) means that his equation has no solutions. So

\[
3x^{3}-2x^{2}+7x+8
\]
		is not a linear combination of the two equations.

\end{example}


\begin{definition}
		Let \(S\) be a nonempty subset of a vector space \(V\). The \textit{span}\index{Span} of \(S\) denoted by span\((S)\), is the set consisting of all linear combinations of the vectors in S.

\end{definition}

	Observe that in \(\mathbb{R}^{3}\), the spanning set of \(\{(1,0,0),(0,1,0)\}\) consists of all vectors in \(\mathbb{R}^{3}\) of the form \(a(1,0,0)+b(0,1,0)\) (by the definition of linear combinations) which is equivalent to \((a,b,0)\) or a point on the \(xy\) plane. Thus, this spanning set contains all points of the \(xy\) plane and is a subset of \(\mathbb{R}\).


\begin{theorem}[Span is a Subspace]
		The span of any subset \(S\) of a vector space \(V\) is a subspace of \(V\) that contains \(S\). Moreover, any subspace of \(V\) that contains \(S\) must also contain the span of \(S\).


\begin{proof}[Incomplete Proof]
			If \(S = \emptyset\), the proof is immediate since the span\((\emptyset) = \{0\}\) and \(\{0\}\) is a subspace of any subspace \(V\) (and in this case, \(\{0\}\) contains \(S\) since any set contains the empty set). Let \(S\) be a nonempty subset of \(V\). Then \(S\) contains a vector, call it \(z\). Then \(0z = 0\) so by the definition of span, \(0\in\) span\((S)\). Let \(x,y\in\) span\((S)\). Then there exists vectors \(u_{1},u_{2},\dots,u_{m}\), \(v_{1},v_{2},\dotsv_{n}\in S\) and scalars \(a_{1},a_{2},\dots,a_{m}\), \(b_{1},b_{2},\dotsb_{n}\in V\) such that

\[
x = a_{1}u_{1} + a_{2}u_{2} +\dots+ a_{m}u_{m} \text{and} y = b_{1}v_{1} + b_{2}v_{2} +\dots+ b_{n}v_{n}
\]
			Then

\[
x + y = a_{1}u_{1} + a_{2}u_{2} +\dots+ a_{m}u_{m} + b_{1}v_{1}+b_{2}v_{2} +\dots+ b_{n}v_{n}
\]
			is a linear combination of vectors in \(S\) so \(x + y \in\) span\((S)\). Similarly,

\[
cx = ca_{1}u_{1} + ca_{2}u_{2} +\dots+ ca_{m}u_{m}
\]
			which is also in span\((S)\). Thus span\((S)\) contains the zero vector, is closed under addition and scalar multiplication, so it is a subspace of V.

\end{proof}

\end{theorem}


\begin{definition}
		A subset \(S\) of a vector space \(V\) \textit{generates}\index{generates} or \textit{spans}\index{spanning set} \(V\) if span\((S) = V\).

\end{definition}


\begin{example}
		The vectors

\[
(1, 1, 0), (1, 0, 1) \text{and} (0, 1, 1)
\]
		span \(\mathbb{R}^{3}\) since any arbitrary vector in \(\mathbb{R}^{3}\) is a linear combination of these 3 vectors. It is possible to compute the scalars \(r,s,t\) which solve the equation

\[
r(1, 1, 0) + s(1, 0, 1) + t(0, 1, 1) = (a_{1}, a_{2}, a_{3})
\]
		by

\[
r = \frac{1}{2}(a_{1} + a_{2} - a_{3}), s = \frac{1}{2}(a_{1} - a_{2} + a_{3}), t = \frac{1}{2}(-a_{1} + a_{2} + a_{3})
\]

\end{example}


\begin{example}
		The polynomials

\[
x^{2} + 3x - 2, 2x^{2} + 5x - 3 \text{and} -x^{2} - 4x + 4
\]
		span or generate \(P_{2}(R)\) since each polynomial in \(P_{2}(R)\) is a linear combination of these 3 polynomials, and they are all in \(P_{2}(R)\).

\end{example}

\subsection{Floating Point Representation}
	A note must be made about the limits of computing large values. The following definition gives limits to the size of values that computers can handle precisely.

	
\begin{definition}
		The \textit{maximum} and \textit{minimum}\index{Floating-Point Limit Values} value for single-precision floating point values are given by
			
\[
\max(\text{singleFloat}) = 3.4\times 10^{38}
\]
			
\[
\min(\text{singleFloat}) = 1.2\times 10^{-38}
\]

		The maximum and minimum value for double-precision floating point values are given by
			
\[
\max(\text{singleFloat}) = 1.8\times 10^{308}
\]
			
\[
\min(\text{singleFloat}) = 2.2\times 10^{-308}
\]
	
\end{definition}

	
\begin{example}
		Using double-precision floating point values, the first example causes an underflow
			
\[
1 + 1\times 10^{-308} = 1 + 0 = 1
\]
			
\[
1 + 3\times 10^{-308} \not = 1
\]
	
\end{example}

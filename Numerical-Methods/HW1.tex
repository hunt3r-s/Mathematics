\subsection*{Homework 1}
\begin{problem}
  Solve the following linear system with five digits of rounding precision
  \[
\systeme*{0.1036x + 0.2122y = 0.7381, 0.2081x + 0.4247y = 0.9327}
\]
\end{problem}

\begin{solution}
  Begin with the system
  \[
\systeme*{0.1036x + 0.2122y = 0.7381, 0.2081x + 0.4247y = 0.9327}
\]
  Seeking an equation of 1 variable, we multiply the second equation by a
  multiple of the first to eliminate \(x\). In this case, the multiple is
  \(\frac{0.2081}{0.1036}\), which gives us
  \[
\systeme[x]{0.1036x + 0.2122y = 0.7381, 0.4247y -
      \frac{0.2081}{0.1036}0.2122y = 0.9327 - \frac{0.2081}{0.1036}0.7381}
\]

  from which we solve the first equation for \(y\) and substitute it into the
  second to obtain
  \[
0.4247y - 2.0087 \times 0.2122y = 0.9327 - 2.0087 \times 0.7381
\]
  \[
-0.0015461y = -0.54992
\]
  \[
y = 355.68
\]
\end{solution}

\begin{problem}
  Solve the linear system by Gaussian Elimination without pivoting, using the
  matrix form
  \[
\systeme*{x_{1} + 5x_{2} + 2x_{3} = 1, 3x_{1} + 6x_{2} + 2x_{3} = -1,
      -2x_{1} + 2x_{2} + x_{3} = 3}
\]
\end{problem}

\begin{solution}
  We begin by finding the lower-upper decomposition of the matrix
  \[
A \vec{x} =
\begin{pmatrix}
                   1 & 5 & 2\\
                   3 & 6 & 2\\
                   -2 & 2 & 1
                 \end{pmatrix}
                 \begin{pmatrix}
                   x_{1}\\
                   x_{2}\\
                   x_{3}
                 \end{pmatrix}
\]
  First, we take \(R_{2} = R_{2} - 3R_{1}\) to obtain
  \[
\begin{pmatrix}
       1 & 5 & 2\\
       0 & -9 & -4\\
       -2 & 2 & 1
     \end{pmatrix}
\]
  Then we take \(R_{3} = R_{3} + 2R_{1}\) to obtain
  \[
\begin{pmatrix}
       1 & 5 & 2\\
       0 & -9 & -4\\
       0 & 12 & 5
     \end{pmatrix}
\]
  Next, we have \(R_{3} = R_{3} + \frac{4R_{2}}{3}\) which gives us
  \[
U =
    \begin{pmatrix}
      1 & 5 & 2\\
      0 & -9 & -4\\
      0 & 0 & -\frac{1}{3}
    \end{pmatrix}
\]
  The \(L\) matrix is given by the coeffiecients of \(A\) corresponding to
  the zero-entries of \(U\)
  \[
\begin{pmatrix}
      1 & 0 & 0\\
      3 & 1 & 0\\
      -2 & -\frac{4}{3} & 1
    \end{pmatrix}
\]
  To solve the system, we need to solve
  \[
LU \vec{x} =
\begin{pmatrix}
                    1 & 0 & 0\\
                    3 & 1 & 0\\
                    -2 & -\frac{4}{3} & 1
                  \end{pmatrix}
                  \begin{pmatrix}
                    1 & 5 & 2\\
                    0 & -9 & -4\\
                    0 & 0 & -\frac{1}{3}
                  \end{pmatrix}
                  \begin{pmatrix}
                    x_{1}\\
                    x_{2}\\
                    x_{3}
                  \end{pmatrix}
=
                  \begin{pmatrix}
                    1\\
                    -1\\
                    3
                  \end{pmatrix}
\]
  Let
  \[
\begin{pmatrix}
       1 & 5 & 2\\
       0 & -9 & -4\\
       0 & 0 & -\frac{1}{3}
     \end{pmatrix}
     \begin{pmatrix}
       x_{1}\\
       x_{2}\\
       x_{3}
     \end{pmatrix}
=
     \begin{pmatrix}
       y_{1}\\
       y_{2}\\
       y_{3}
     \end{pmatrix}
\]
   So that
   \[
\begin{pmatrix}
        1 & 0 & 0\\
        3 & 1 & 0\\
        -2 & -\frac{4}{3} & 1
      \end{pmatrix}
      \begin{pmatrix}
        y_{1}\\
        y_{2}\\
        y_{3}
      \end{pmatrix}
=
      \begin{pmatrix}
        1\\
        -1\\
        3
      \end{pmatrix}
\]
    From this we conclude that
    \[
y_{1} = 1 \implies y_{2} = -1 - 3(1) = -4
      \implies y_{3} = 3 + 2(1) + (\frac{4}{3})(-4)
      = -\frac{1}{3}
\]
Now let
    \[
\begin{pmatrix}
         1 & 5 & 2\\
         0 & -9 & -4\\
         0 & 0 & -\frac{1}{3}
       \end{pmatrix}
       \begin{pmatrix}
         x_{1}\\
         x_{2}\\
         x_{3}
       \end{pmatrix}
=
       \begin{pmatrix}
         y_{1}\\
         y_{2}\\
         y_{3}
       \end{pmatrix}
=
       \begin{pmatrix}
         1\\
         -4\\
         -\frac{1}{3}
       \end{pmatrix}
\]
    So that
    \[
x_{3} = -1 \implies -9x_{2} - 4(1) = -4
      \implies x_{2} = 0
\]
    \[
\implies x_{1} = 1 + 2(-1) = -1 \implies
      x_{1} = -1
\]
    The values
    \(x_{1} = -1, x_{2} = 0, x_{3} = 1\) correspond
    to the solutions to the original system.
  \end{solution}

\begin{problem}
  Solve the linear system by Gaussian Elimination with partial pivoting, using the
  matrix form
  \[
\systeme*{x_{1} + 5x_{2} + 2x_{3} = 1, 3x_{1} + 6x_{2} + 2x_{3} = -1,
      -2x_{1} + 2x_{2} + x_{3} = 3}
\]
\end{problem}

\begin{solution}

  We begin with the matrix
  \[
A =
\begin{pmatrix}
           1 & 5 & 2\\
           3 & 6 & 2\\
           -2 & 2 & 1
         \end{pmatrix}
\]
  First, we swap \(R_{1}\) and \(R_{2}\) to obtain
  \[
A' =
\begin{pmatrix}
           3 & 6 & 2\\
           1 & 5 & 2\\
           -2 & 2 & 1
          \end{pmatrix}
\]
        This is reflected in our permutation matrix \(P\) by
        \[
P_{1}A =
\begin{pmatrix}
                      0 & 1 & 0\\
                      1 & 0 & 0\\
                      0 & 0 & 1
                    \end{pmatrix}
                    \begin{pmatrix}
                      1 & 5 & 2\\
                      3 & 6 & 2\\
                      -2 & 2 & 1
                    \end{pmatrix}
=
                    \begin{pmatrix}
                      3 & 6 & 2\\
                      1 & 5 & 2\\
                      -2 & 2 & 1
                    \end{pmatrix}
\]
  Next, we have \(R_{2} = -\frac{1}{3}R_{1}+R_{2}\) and \(R_{3} = -\frac{2}{3}R_{1}
  + R_{3}\) from which the product becomes
  \[
P_{2}A =
\begin{pmatrix}
                 1 & 0 & 0\\
                 \frac{1}{3} & 1 & 0\\
                 \frac{2}{3} & 0 & 1
               \end{pmatrix}
               \begin{pmatrix}
                 3 & 6 & 2\\
                 1 & 5 & 2\\
                 -2 & 2 & 1
               \end{pmatrix}
=
               \begin{pmatrix}
                 1 & 5 & 2\\
                 0 & 3 & \frac{4}{3}\\
                 0 & 6 & \frac{7}{3}
               \end{pmatrix}
\]
  Now, \(R_{3} = 2R_{2} - R_{3}\) gives
  \[
U =
\begin{pmatrix}
            1 & 0 & 0\\
            0 & 0 & 1\\
            0 & 1 & -\frac{1}{2}
          \end{pmatrix}
          \begin{pmatrix}
            1 & 5 & 2\\
            0 & 6 & \frac{7}{3}\\
            0 & 0 & \frac{1}{6}
          \end{pmatrix}
=
        \begin{pmatrix}
          3 & 6 & 2\\
          0 & 3 & \frac{4}{3}\\
          0 & 0 & \frac{1}{3}
        \end{pmatrix}
\]
  Thus we get the relation
  \[
\begin{pmatrix}
       1 & 0 & 0\\
       0 & 0 & 1\\
       0 & 1 & -\frac{1}{2}
     \end{pmatrix}
     \begin{pmatrix}
       1 & 5 & 2\\
       3 & 6 & 2\\
       -2 & 2 & 1
     \end{pmatrix}
=
     \begin{pmatrix}
       1 & 0 & 0\\
       \frac{1}{2} & 1 & 0\\
       -2 & -\frac{1}{3} & 1
     \end{pmatrix}
     \begin{pmatrix}
       3 & 6 & 2\\
       0 & 3 & \frac{4}{3}\\
       0 & 0 & \frac{1}{3}
     \end{pmatrix}
\]
   As found previously, this has solutions \(x_{1} = -1, x_{2} = 0, x_{3} = 1\).
  Alternatively, the problem can be
  solved in Python
    \inputminted[frame = lines, bgcolor = lightgray, linenos]{python}{PythonSources/LUfactor.py}
\end{solution}
%%% Local Variables:
%%% mode: latex
%%% TeX-master: "NumericalMethodsMain"
%%% End:

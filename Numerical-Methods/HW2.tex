\subsection*{Homework 2}
\begin{problem}
  Prove that the matrix
  \(
\begin{pmatrix}
    2 & 1\\
    1 & 2
  \end{pmatrix}
\) is symmetric and positive definite.
\end{problem}

\begin{solution}
  Let \(A =
\begin{pmatrix}
    2 & 1\\
    1 & 2
  \end{pmatrix}
\). By swapping the rows and columns of \(A\), we obtain
  \[
A =
\begin{pmatrix}
    2 & 1\\
    1 & 2
  \end{pmatrix}
= A^{t}.
\]
  Since the transpose of \(A\) is \(A\), it is a symmetric matrix. Now, observe that subtracting \(\lambda\) from the diagonals of \(A\), the eigenvalues are given by \(\lambda_{1} = 3\) and \(\lambda_{2} = 1\). Since \(A\) is symmetric and has all positive eigenvalues, \(A\) is a positive definite matrix.
\end{solution}

\begin{problem}

\end{problem}
  Find the Cholesky factorization of
  \(A =
\begin{pmatrix}
    2 & 1\\
    1 & 2
  \end{pmatrix}
\).
\begin{solution}
  Consider the lower diagonal matrix given by
  \[
L =
  \begin{pmatrix}
    \sqrt{2} & 0\\
    \frac{1}{\sqrt{2}} & \sqrt{\frac{3}{2}}
  \end{pmatrix}
,
\]
  whose diagonal is positive and whose transpose is given by
  \[
L^{T} =
  \begin{pmatrix}
   \sqrt{2} & \frac{1}{\sqrt{2}}\\
   0 & \sqrt{\frac{3}{2}}
  \end{pmatrix}
.
\]
  We have
  \[
LL^{T} =
  \begin{pmatrix}
    \sqrt{2} & 0\\
    \frac{1}{\sqrt{2}} & \sqrt{\frac{3}{2}}
  \end{pmatrix}
  \begin{pmatrix}
   \sqrt{2} & \frac{1}{\sqrt{2}}\\
   0 & \sqrt{\frac{3}{2}}
  \end{pmatrix}
=
  \begin{pmatrix}
    2 & 1\\
    1 & 2
  \end{pmatrix}
=
  A,
\]
  which is in the form of a Cholesky factorization. This can also be solved in Python:
  \inputminted[frame = lines, bgcolor = lightgray, linenos]{python}{PythonSources/cholesky.py}
\end{solution}

\begin{problem}
  For the matrix
  \[
A =
  \begin{pmatrix}
   -1 & 1 & 0\\
   -4 & 3 & 0\\
   1 & 0 & 2
  \end{pmatrix}
,
\]
  find all the eigenvalues and their corresponding eigenspaces. Additionally, find the algebraic and geometric multiplicity of each eigenvalue.
\end{problem}

\begin{solution}
  We begin by finding the eigenvalues. Seeking the characteristic polynomial, we compute \(A-\lambda I\). Now,
  \[
A-\lambda I =
\begin{pmatrix}
   -\lambda -1 & 1 & 0\\
   -4 & 3-\lambda & 0\\
   1 & 0 & 2
  \end{pmatrix}
\]
  Now we need to find \(\lambda\) such that \(\det(A-\lambda I) = \vec{0}\),
  \[
\det(A - \lambda I) = (0)(-1)^{4}
  \begin{vmatrix}
    -4 & 3-\lambda\\
    1 & 0
  \end{vmatrix}
+
  0(-1)^{5}
  \begin{vmatrix}
    -\lambda-1 & 1\\
    1 & 0
  \end{vmatrix}
  + (2-\lambda)(-1)^{6}
  \begin{vmatrix}
    -\lambda-1 & 1\\
    -4 & 3-\lambda
  \end{vmatrix}
\]
  \[
= (2-\lambda)
  \begin{vmatrix}
    -\lambda-1 & 1\\
    -4 & 3-\lambda
  \end{vmatrix}
=
  (-\lambda-1)(3-\lambda)-1(-4) = \lambda^{2} - 2\lambda + 1 = -(\lambda-2)(\lambda-1)^{2}.
\]
  The roots of the characteristic polynomial yield the eigenvalues for \(A\) so \(\lambda_{1} = 2, \lambda_{2} = 1\). The term \((\lambda-1)^{2}\) indicates that the eigenvalue \(\lambda_{2}\) has an algebraic multiplicity of 2, by similar argument, \(\lambda_{1}\) has an algebraic multiplicity of 1. Substituting our eigenvalues into \(A-\lambda I\), we get
  \[
A-2I =
  \begin{pmatrix}
    -3 & 1 & 0\\
    -4 & 1 & 0\\
    1 & 0 & 0
  \end{pmatrix}
,
\]
  which has a null space of
  \(
\begin{pmatrix}
    0 \\
    0 \\
    1
  \end{pmatrix}
\)
  and for the eigenvalue 1,
  \[
A-1I =
  \begin{pmatrix}
    -2 & 1 & 0\\
    -4 & 2 & 0\\
    1 & 0 & 1
  \end{pmatrix}
,
\]
  which has a null space of
   \(
\begin{pmatrix}
    -1\\
    -2\\
    1
  \end{pmatrix}
\). Since these vectors consist of the null space of \(A-\lambda I\), these are the eigenvectors of their associated eigenvalues. Since both null spaces are 1-dimensional, the geometric multiplicity of both \(\lambda_{1}\) and \(\lambda_{2}\) is 1.

  This can be done in Python with the following:

  \inputminted[frame = lines, bgcolor = lightgray, linenos]{python}{PythonSources/eigenvals.py}

\end{solution}



%%% Local Variables:
%%% mode: latex
%%% TeX-master: "NumericalMethodsMain"
%%% End:

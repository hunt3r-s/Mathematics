\subsection{Matrix Factorization}
A system of linear equations can be written in matrix form
\[
A\vec{x} = \vec{b}
\]

The main objective is to show that Gaussian Elimination applied to \(A\) yields a factorization of \(A\) into a product of a lower triangular and an upper triangular matrix. Recall that a lower triangular matrix is a matrix whose entries consist of all 1's along the diagonal and 0's in the upper half, an upper triangular matrix has all non-zero entries on or above the main diagonal.
\begin{example}
  Consider the system of equations represented in matrix form
  \[
  \begin{pmatrix}
    6 & -2 & 2 & 4\\
    12 & -8 & 6 & 10\\
    3 & -13 & 9 & 3\\
    -6 & 4 & 1 & -18
  \end{pmatrix}
  \begin{pmatrix}
    x_{1}\\
    x_{2}\\
    x_{3}\\
    x_{4}
  \end{pmatrix}
=
  \begin{pmatrix}
    16\\
    26\\
    -19\\
    -34
  \end{pmatrix}
  \]
  After the first step of Gaussian Elimination, we get the system
  \[
  \begin{pmatrix}
    6 & -2 & 2 & 4\\
    0 & -4 & 2 & 2\\
    0 & -12 & 8 & 1\\
    0 & 2 & 3 & -14
  \end{pmatrix}
  \begin{pmatrix}
    x_{1}\\
    x_{2}\\
    x_{3}\\
    x_{4}
  \end{pmatrix}
=
  \begin{pmatrix}
    16\\
    -6\\
    -27\\
    -18
  \end{pmatrix}
  \]
  Notice that this form can be obtained by multiplying \(A\) with a lower triangular matrix \(M_{1}\) as follows,
  \[
M_{1} A \vec{x} = M_{1} \vec{b}
\]
  where
  \[
  M_{1} A =
  \begin{pmatrix}
    1 & 0 & 0 & 0\\
    -2 & 1 & 0 & 0\\
    -\frac{1}{2} & 0 & 1 & 0\\
    1 & 0 & 0 & 1
  \end{pmatrix}
  \begin{pmatrix}
    6 & -2 & 2 & 4\\
    12 & -8 & 6 & 10\\
    3 & -13 & 9 & 3\\
    -6 & 4 & 1 & -18
  \end{pmatrix}
=
  \begin{pmatrix}
    6 & -2 & 2 & 4\\
    0 & -4 & 2 & 2\\
    0 & -12 & 8 & 1\\
    0 & 2 & 3 & -14
  \end{pmatrix}
  \]
  After performing two more steps of Gaussian elimination, we are left with the system
  \[
  M_{3}M_{2}M_{1}A \vec{x} =
  \begin{pmatrix}
    1 & 0 & 0 & 0\\
    0 & 1 & 0 & 0\\
    0 & 0 & 1 & 0\\
    0 & 0 & -2 & 1
  \end{pmatrix}
  \begin{pmatrix}
    1 & 0 & 0 & 0\\
    0 & 1 & 0 & 0\\
    0 & -3 & 1 & 0\\
    0 & \frac{1}{2} & 0 & 1
  \end{pmatrix}
  \begin{pmatrix}
    1 & 0 & 0 & 0\\
    -2 & 1 & 0 & 0\\
   -\frac{1}{2} & 0 & 1 & 0\\
   1 & 0 & 0 & 1
  \end{pmatrix}
=
  \begin{pmatrix}
    6 & -2 & 2 & 4\\
    0 & -4 & 2 & 2\\
    0 & 0 & 2 & -5\\
    0 & 0 & 0 & -3
  \end{pmatrix}
  = U
  \]
  Finally, we can obtain a representation of \(A\) in terms of a lower-diagonal matrix \(L\) and upper diagonal \(U\)
  \[
A = M^{-1}U = M_{1}^{-1}M_{2}^{-1}M_{3}^{-1}U = LU
\]
  where
  \[
L =
  \begin{pmatrix}
    1 & 0 & 0 & 0\\
    2 & 1 & 0 & 0\\
    \frac{1}{2} & 0 & 1 & 0\\
    -1 & 0 & 0 & 1
  \end{pmatrix}
  \begin{pmatrix}
    1 & 0 & 0 & 0\\
    0 & 1 & 0 & 0\\
    0 & 3 & 1 & 0\\
    0 & -\frac{1}{2} & 0 & 1
  \end{pmatrix}
  \begin{pmatrix}
    1 & 0 & 0 & 0\\
    0 & 1 & 0 & 0\\
    0 & 0 & 1 & 0\\
    0 & 0 & 2 & 1
  \end{pmatrix}
  \]
\end{example}

\begin{theorem}
  Let \(A = (a_{ij})\) be an \(n \times n\) matrix, assume the forward elimination phase of the naive Gaussian algorithm is applied to \(A\) without encountering any 0 divisiors. Let the matrix resulting from this forward elimination phase be denoted by \(\tilde{A} = \tilde{a_{ij}}\). If \(L\) is a lower diagonal matrix whose non-zero entries correspond to \(\tilde{a_{ij}}\) and \(U\) is an upper diagonal matrix whose non-zero entries correspond to \(\tilde{a_{ij}}\), then \(A = LU\)
\end{theorem}

For a symmetrix matrix that has an ordinary \(LU\) factorization, there is another way to factorize the matrix. We can obtain a \(LDL^{T}\) matrix where \(D\) is a diagonal matrix. Notice that for a symmetric matrix
\[
LU = A = A^{T} = (LU)^{T} = U^{T}L^{T} = LDL^{T}
\]

Using Python, one can easily find the \(LDL^{T}\) factorizaiton of a matrix:

\inputminted[frame = lines, bgcolor = lightgray, linenos]{python}{PythonSources/ldlt.py}

\begin{example}
  To find the \(LDL^{T}\) factorization for the matrix
  \[
A =
  \begin{pmatrix}
    4 & 3 & 2 & 1\\
    3 & 3 & 2 & 1\\
    2 & 2 & 2 & 1\\
    1 & 1 & 1 & 1
  \end{pmatrix}
  \]
  we first find the \(LU\) factorization which is given by
  \[
A = LU =
    \begin{pmatrix}
      1 & 0 & 0 & 0\\
      \frac{3}{4} & 1 & 0 & 0\\
      \frac{1}{2} & \frac{2}{3} & 1 & 0\\
      \frac{1}{4} & \frac{1}{3} & \frac{1}{2} & 1
    \end{pmatrix}
    \begin{pmatrix}
      4 & 3 & 2 & 1\\
      0 & \frac{3}{4} & \frac{1}{2} & \frac{1}{4}\\
      0 & 0 & \frac{2}{3} & \frac{1}{3}\\
      0 & 0 & 0 & \frac{1}{2}
    \end{pmatrix}
    \]
    Then, we extract the diagonal elements of \(U\) and put them in \(D\) to obtain
    \[
U =
    \begin{pmatrix}
      4 & 0 & 0 & 0\\
      0 & \frac{3}{4} & 0 & 0\\
      0 & 0 & \frac{2}{3} & 0\\
      0 & 0 & 0 & \frac{1}{2}
    \end{pmatrix}
    \begin{pmatrix}
      1 & \frac{3}{4} & \frac{1}{2} & \frac{1}{4}\\
      0 & 1 &\frac{2}{3} & \frac{1}{3} \\
      0 & 0 & 1 & \frac{1}{2}\\
      0 & 0 & 0 & 1
    \end{pmatrix}
= DL^{T}
\]
\end{example}
When a matrix is symmetric \textit{and} positive definite (a matrix is positive definite if \(\vec{x}^{T}A \vec{x}>0\) for every non zero \(\vec{x}\)), we can further simplify the factorization into a form called the \textit{Cholesky factorization}\index{Cholesky factorization}

\begin{theorem}
  If A is a real, symmetric positive definite matrix, then it has a unique factorization \(A = LL^{T}\) where \(L\) is lower triangular and has a positive diagonal.
\end{theorem}


\begin{example}
  The matrix
  \[ \begin{pmatrix}
       4 & 3 & 2 & 1\\
       3 & 3 & 2 & 1\\
       2 & 2 & 2 & 1\\
       1 & 1 & 1 & 1
     \end{pmatrix} \]
   has a Cholesky factorization of the form
   \[ A = LDL^{T} = (LD^{\frac{1}{2}})(D^{\frac{1}{2}}L^{T}) = \tilde{L}\tilde{L^{T}} \]
   where
   \[ \begin{pmatrix}
        1 & 0 & 0 & 0\\
        \frac{3}{4} & 1 & 0 & 0\\
        \frac{1}{2} & \frac{2}{3} & 1 & 0\\
        \frac{1}{4} & \frac{1}{3} & \frac{1}{2} & 1
      \end{pmatrix}
      \begin{pmatrix}
        2 & 0 & 0 & 0\\
        0 & \frac{1}{2}\sqrt{3} & 0 & 0\\
        0 & 0 & \sqrt{\frac{2}{3}} & 0\\
        0 & 0 & 0 & \frac{1}{\sqrt{2}}
      \end{pmatrix} = 
      \begin{pmatrix}
        2 & 0 & 0 & 0\\
        \frac{3}{2} & \frac{1}{2}\sqrt{3} & 0 & 0\\
        1 & \frac{1}{3}\sqrt{3} & \sqrt{\frac{2}{3}} & 0\\
        \frac{1}{2} & \frac{1}{6}\sqrt{3} & \frac{1}{2}\sqrt{\frac{2}{3}} & \frac{1}{\sqrt{2}}
      \end{pmatrix} \]
  \end{example}
  
  \begin{theorem}
    If a matrix is symmetric and has all positive eigenvalues if and only if it is positive definite.
    \begin{proof}
      Let \(A\) be a positive definite matrix. Let zero be an eigenvalue, then there exists an eigenvector \(x\) such that \(Ax = 0.\) But then \(x^{T}Ax = 0\) so \(A\) is not positive definite. Let \(\lambda\) be an eigenvalue such that \(\lambda<0\), then there is a vector \(x\) such that \(Ax = \lambda x\) where \(x^{T}Ax = \lambda\abs{x^{2}}\). But \(\lambda\abs{x^{2}}\) is negative since \(\lambda\) is negative and \(\abs{x^{2}}>0\) so \(A\) is not positive definite. By contradiction, if a matrix is positive definite, it has all positive eigenvalues.\\
      Let \(A\) be a symmetric matrix with all positve eigenvalues \(\lambda\). Then there exists a matrix factorization such that \(A = QDQ^{T}\) where \(D\) is a diagonal matrix. Then we have
      \[ x^{T}Ax = x^{T}(Q^{T}DQ)x = (x^{T}Q^{T})D(Qx) = (Qx)^{T}D(Qx) = \sum^{n}_{i=1}\lambda_{i}(q_{i}x_{i}^{2}) > 0\]
      Thus \(A\) is a positive symmetric matrix.
    \end{proof}
  \end{theorem}
  
  We can implement this theorem in Python to check if a matrix is symmetric, and then positive definite. If so, it can be factorized in Cholesky form.
  
  \inputminted[frame = lines, bgcolor = lightgray, linenos]{python}{PythonSources/spdCheck.py}

%%% Local Variables:
%%% mode: latex
%%% TeX-master: "NumericalMethodsMain"
%%% End:


\subsection{Significant Digits, Rounding and Chopping}

	
\begin{definition}
		\textit{Significant digits}\index{Significant Digits} are digits beginning with the leftmost nonzero digit and ending with the rightmost correct digit
	
\end{definition}

	Significant digits are a representation of the number of digits of precision that a measurement has.

	
\begin{definition}
		\textit{Normalized Scientific Notation}\index{Normalized Scientific Notation} for the decimal system is given by
			
\[
x = \pm 0.d_{1}d_{2}d_{3}\dots\times 10^{n}
\]
		where \(d_{1} \not = 0\)and \(n\) is an integer.
	
\end{definition}

	
\begin{example}
		The following is in normalized scientific notation for decimals
			
\[
0.12345 \times 10^{5}
\]
		The following are not proper representations
			
\[
0.012345 \times 10^{6}
\]
			
\[
1.2345 \times 10^{4}	
\]
	
\end{example}

	
\begin{definition}
		Normalized Scientific Notation for the binary system is given by
			
\[
x = \pm 0.1d_{1}d_{2}d_{3}\dots\times 2^{\pm n}
\]
		where \(d_{i} = 0\) or \(d_{i} = 1\) and \(n\) is an integer.
	
\end{definition}

	
\begin{example}
		The following is in normalized scientific notation for binary
			
\[
0.101 \times 2^{-2}
\]
		The following are not proper representations
			
\[
0.0101\times 2^{-1}
\]
			
\[
1.01\times 2^{-3}
\]
		
\end{example}

	Scientific notation may have other representations, but they are equivalent to the above definitions.

	
\begin{definition}
		Suppose \(\alpha\) and \(\beta\) are two numbers where one is an approximation of the other. The error of \(\beta\) as an approximation to \(\alpha\) is \(\alpha - \beta\). The \textit{absolute error}\index{absolute error} is the absolute value \(|\alpha-\beta|\). The \textit{relative error}\index{relative error} of \(\beta\) as an approximation to \(\alpha\) is \(\frac{|\alpha-\beta}{\alpha}\).
	
\end{definition}

	The distinction between the two errors is that absolute error allows \(\alpha\) and \(\beta\) to be interchanged while relative error requires one of the values to be specified as correct.

	
\begin{example}
		Consider x = 0.00347 rounded to x' = 0.0035 and y = 30.158 rounded to \(0.3\times 10^{-4}\)\\
		The \(x' = 0.35\times 10^{-2}\) has two significant digits with absolute error of \(0.3\times 10^{-4}\), relative error of \(0.865\times 10^{-2}\) and \(y' = 0.3015\times 10^{2}\) has four significant digits with absolute error of \(0.2\times 10^{-2}\) and relative error of \(0.66\times 10^{-4}\)
	
\end{example}

In Python, the following code will return the amount of significant digits in a number
\inputminted[frame = lines, bgcolor = lightgray, linenos]{python}{PythonSources/sigfigs.py}
	
\begin{definition}
		A number is \textit{chopped}\index{chopped} to n digits if all the digits following the n'th are removed from the end of the number.
	
\end{definition}

	
\begin{example}
		The following numbers rounded to two decimal places
			
\[
0.217 \approx 0.22,\; 0.475\approx 0.48,\; 0.592\approx 0.59
\]
		Chopping them to two significant digits gives
			
\[
0.217 \approx 0.21,\; 0.475\approx 0.47,\; 0.592\approx 0.59
\]
	
\end{example}

Chopping and rounding numbers is very straightforward in Python:
\inputminted[frame = lines, bgcolor = lightgray, linenos]{python}{PythonSources/chopRound.py}

\begin{example}
		Where \(n\) denotes the number of significant digits,
			
\[
0.031,\; n = 2
\]
			
\[
031,\; n = 2
\]
			
\[
3.1400,\; n = 5
\]
			
\[
100.,\; n = 3
\]
			
\[
100.0,\; n = 4
\]
			
\[
100,\; n = 1
\]
			
\[
1230,\; n = 3
\]
	
\end{example}

	
\begin{example}
		Solve the following linear system by carrying out three significant digits of rounding precision and then four.
			
\[
\systeme*{0.1036x + 0.2122y = 0.7381, 0.2081x + 0.4247y = 0.9327}
\]

		When solving to 3 significant digits, we begin with the linear system
			
\[
\systeme*{0.104x + 0.212y = 0.738, 0.208x + 0.425y = 0.933}
\]

		Then we take choose multiple of the first equation such that the coefficient of \(x\) becomes 0. In this case, \(\frac{0.208}{0.104}\)
			
\[
\systeme[x]{0.104x + 0.212y = 0.738, 0.425y - \frac{0.208}{0.104}0.212y = 0.933 - \frac{0.208}{0.104}0.738}
\]
		from which we get
			
\[
0.425y - 0.424y = 0.933 - 1.48
\]
			
\[
0.001y = -0.547
\]
			
\[
y = -547
\]
		With four significant digits we have
			
\[
\systeme*{0.1036x + 0.2122y = 0.7381, 0.2081x + 0.4247y = 0.9327}
\]
		Then we choose a multiple of the first equation such that the coefficient of \(x\) becomes 0. In this case, \(\frac{0.2081}{0.1036}\)
			
\[
\systeme[x]{0.1036x + 0.2122y = 0.7381, 0.4247y - \frac{0.2081}{0.1036}0.2122y = 0.9327 - \frac{0.2081}{0.1036}0.7381}
\]
		and solving for \(y\) yields
		from which we get
			
\[
0.4247y - 2.009 \times 0.2122y = 0.9327 - 2.009 \times 0.7381
\]
			
\[
-0.001600y = -0.5503
\]
			
\[
y = 343.9
\]
	
\end{example}

	From the above example, we should observe that accurate data should have its calculations carried out to full precision, otherwise the result will be inaccurate.
%%% Local Variables:
%%% mode: latex
%%% TeX-master: "NumericalMethodsMain"
%%% End:
